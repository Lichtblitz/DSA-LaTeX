\documentclass{dsa}

\pagenumbering{gobble}

\usepackage{multicol}

\usepackage{listings}
\definecolor{mygreen}{rgb}{0,0.6,0}
\definecolor{mygray}{rgb}{0.5,0.5,0.5}
\definecolor{mymauve}{rgb}{0.58,0,0.82}

\lstset{ %
  basicstyle=\footnotesize\ttfamily, % the size of the fonts that are used for the code
  breakatwhitespace=false,           % sets if automatic breaks should only happen at whitespace
  breaklines=true,                   % sets automatic line breaking
  captionpos=b,                      % sets the caption-position to bottom
  commentstyle=\color{mygreen},      % comment style
  keepspaces=true,                   % keeps spaces in text, useful for keeping indentation of code (possibly needs columns=flexible)
  keywordstyle=\color{blue},         % keyword style
  language=[LaTeX]TeX,               % the language of the code
  numberstyle=\tiny\color{mygray},   % the style that is used for the line-numbers
  showspaces=false,                  % show spaces everywhere adding particular underscores; it overrides 'showstringspaces'
  showstringspaces=false,            % underline spaces within strings only
  showtabs=false,                    % show tabs within strings adding particular underscores
  stringstyle=\color{mymauve},       % string literal style
  tabsize=2,                         % sets default tabsize to 2 spaces
}

\sloppy

\begin{document}
   \begin{dsaTitlePage}
      \bfseries \Huge \LaTeX-Klasse für Das Schwarze Auge \\[15pt]
      \LARGE Für Charakterbögen, Abenteuer und Sonstiges \\[40pt]
      {\large \normalfont Felix Krause \\[10pt] \today \\[10pt] Version 0.2} \\[15pt]
      {\small \normalfont Diese Dokumentation ist lizensiert unter den Bedingungen der
       LaTeX Project Public License, Version 1.3c oder neuer.}
   \end{dsaTitlePage}
      
   \twocolumn[\section*{Überblick}]
   
   \subsection*{Einleitung}
   
   Dies ist eine \LaTeX-Klasse, die Umgebungen und Kommandos zur Verfügung
   stellt, um Dokumente im Stil der DSA-Regelbücher und Charakterbögen zu 
   erstellen - auf der Basis der Grafiken aus dem DSA-Fanpaket.
   
   \subsection*{Abhängigkeiten}
   
   Die Schriftart der Überschriften heißt \textit{Mason} und ist
   kostenpflichtig. Es gibt jedoch einen Klon namens \textit{Manson}, der
   kostenlos verfügbar ist. Installiere zumindest die Variante
   \textit{Manson Bold} auf deinem System (auf Windows und OSX kannst du auf
   die heruntergeladene Schriftart doppelklicken und dort auf
   \textit{installieren} klicken). Für normale Schrift wird die freie Schriftart
   \textit{GaramondNo8} verwendet, die du \href{http://garamond.org}{hier}
   herunterladen kannst.
   
   Das Fanpaket, das die Grafiken beinhaltet, wird von der Makefile automatisch
   heruntergeladen (für mehr Informationen zum Makefile, siehe den Abschnitt
   \textit{Benutzung}). Da die Dateinamen Leerzeichen beinhalten, werden sie
   nach dem Herunterladen umbenannt. Falls du Windows benutzt und mit der
   Makefile nichts anfangen kannst, beachte die Anweisungen im Kasten unten.
   
   \vspace{12pt}
   
   \begin{dsaBoxPortrait}
      \subsection*{Windows: Manuelles Setup}
      Lade das Fanpaket
      \href{http://www.ulisses-spiele.de/download/889/}{hier} herunter und
      entpacke es in das Verzeichnis, wo \texttt{dsa.cls} sich befindet. Den
      Ordner, der die Bilder enthält, musst du in \texttt{fanpaket} umbenennen.
      
      Nach dem Entpacken müssen folgende Dateien wiefolgt umbenannt werden:
      
      \scriptsize
      \begin{itemize} \itemsep0em
         \item \texttt{Balken - Seite - links.png} $\Rightarrow$ \texttt{balken-links.png}
         \item \texttt{Balken - Seite - rechts.png} $\Rightarrow$ \texttt{balken-rechts.png}
         \item \texttt{Balken - Seite (Symbole) - links.png} \\ $\Rightarrow$ \texttt{balken-symbole-links.png}
         \item \texttt{Balken - Seite (Symbole) - rechts.png} \\ $\Rightarrow$ \texttt{balken-symbole-rechts.png}
         \item \texttt{Logo - Fanprodukt.png} $\Rightarrow$ \texttt{logo-fanprodukt.png}
         \item \texttt{Logo - Fanprojekt.png} $\Rightarrow$ \texttt{logo-fanprojekt.png}
         \item \texttt{Kasten - Halbseiter.png} \\ $\Rightarrow$ \texttt{kasten-halbseiter.png}
         \item \texttt{Kasten - Viertelseiter.png} \\ $\Rightarrow$ \texttt{kasten-viertelseiter.png}
         \item \texttt{Kasten - Achtelseiter.png} \\ $\Rightarrow$ \texttt{kasten-achtelseiter.png}
      \end{itemize}
   \end{dsaBoxPortrait}
   
   Das Hintergrundbild für Charakterbögen ist nicht Bestandteil des Fanpakets.
   Standardmäßig verwendet die \LaTeX-Klasse statt dessen die Seitenränder mit
   Symbolen für Charakterbögen. Wenn du Charakterbögen mit Hintergrundbild
   erstellen möchtest, lege dein Hintergrundbild in den Ordner \texttt{fanpaket}
   und nenne es \texttt{wallpaper.png}. Das Original-Hintergrundbild lässt sich
   aus der PDF des Kampfprotokolls von \textit{Wege des Schwerts} extrahieren.
   Extrahieren lassen sich Bilder aus einem PDF zum Beispiel mit dem Adobe
   Acrobat - es existieren auch kostenlose Werkzeuge, etwa
   \href{http://sourceforge.net/projects/pdf-images/}{PDF-Images} für OSX.
   Dieses Programm extrahiert die Bilder im \texttt{.ppm}-Format. Dieses kann
   zum Beispiel mit \href{http://www.gimp.org}{GIMP} geladen und als
   \texttt{.png} exportiert werden.
   
   Zum Erstellen der Dokumente wird eine TeX-Distribution benötigt, die
   \texttt{xelatex} beinhaltet (das trifft auf alle derzeit verbreiteten
   Distributionen zu). Außerdem müssen folgende LaTeX-Pakete verfügbar sein:
   \texttt{polyglossia, xunicode, fontspec, titlesec, xcolor, pgf, graphics
    wallpaper, environ}.
   
   Des Weiteren benötigst du \texttt{eforms} für editierbare Textfelder und so.
   Im Gegensatz zu den anderen Paketen musst du dieses manuell
   \href{http://www.ctan.org/pkg/eforms}{hier} herunterladen und dann in deine
   Distribution installieren (wie genau das geht, hängt von der Distribution ab).
   
   \subsection*{Mitgelieferte Dokumente}
   
   Der Klasse liegen bereits einige Dokumente bei, die mit der Makefile oder
   von Hand erzeugt werden können. Zur Erzeugung einfach folgendes Kommando
   eingeben (hier beispielhaft für dieses Dokument):
   
   \hspace{30pt}\texttt{make dokumentation.pdf}
   
   Beim ersten Aufruf wird dieses Kommando das Fanpaket herunterladen. Auf
   Windows kann statt dessen folgendes Kommando verwendet werden, nachdem
   das Fanpaket manuell heruntergeladen wurde:
   
   \hspace{30pt}\texttt{xelatex dokumentation.tex}
   
   Die Dokumente sind im Einzelnen:
   
   \begin{itemize}
      \item \textbf{dokumentation.tex:} Dieses Dokument.
      \item \textbf{vertrautendokument.tex:} Ein Charakterbogen für
            Vertrautentiere von Hexen und Geoden.
   \end{itemize}
   
   \onecolumn
   
   \section*{Benutzung}
   
   \begin{multicols}{2}
   
      \subsection*{Vorbereitung}
      
      Auf OSX und Linux kann das Fanpaket mit folgendem Kommando herundergeladen
      und in einen benutzbaren Zustand gebracht werden:
      
      \hspace{30pt}\texttt{make fanpaket-anpassung}
      
      Auf Windows muss die Anleitung im Kasten auf der vorherigen Seite
      ausgeführt werden. Optional kann nun noch ein \texttt{wallpaper.png}
      in den Ordner \texttt{fanpaket} gelegt werden wie im vorherigen
      Kapitel beschrieben.
      
      \subsection*{Dokumentengerüst}
         
      Ein DSA-Dokument sieht etwa so aus:
      
      \begin{lstlisting}
\documentclass{dsa}
\begin{document}
   \begin{dsaTitlePage}
      \bfseries \Huge
      Mein tolles DSA-Dokument
   \end{dsaTitlePage}
   
   % Inhalt
\end{document}\end{lstlisting}
   
   \columnbreak
      
      Die \texttt{dsaTitlePage} ist für Charakterbögen natürlich nicht notwendig,
      ebensowenig wie die Überschriften. Das Erstellen von Charakterbögen wird
      auf der nächsten Seite beschrieben.
      
      \subsection*{Überschriften \& Layout}

      Überschriften werden, wie man in diesem Dokument sehen kann, in der
      Schriftart \textit{Manson} gesetzt. Standardmäßig wird ein zweispaltiges
      Layout verwendet. Eine Überschrift, die über die gesamte Seitenbreite
      geht, lässt sich folgendermaßen setzen:

      \begin{lstlisting}
\twocolumn[\section*{Überschrift}]\end{lstlisting}

      Dies startet eine neue Seite. Wenn - wie auf dieser Seite - Ein Kasten
      über die gesamte Breite der Seite gehen soll, muss die Seite selbst mit
      \verb|\onecolumn| begonnen werden. Die zweispaltigen Teile der Seite
      können dann mithilfe der \verb|multicols|-Umgebung erstellt werden. 
      
   \end{multicols}
   
   \begin{dsaBoxLandscape}
      \subsection*{Mit Grafiken hinterlegte Boxen}
      
      \begin{multicols}{2}
         Es ist drei Größen für Boxen verfügbar, die mit
         folgenden Umgebungen benutzt werden können:
         
         \begin{itemize} \itemsep0em
            \item \textbf{dsaBoxLandscape}: Querformat, gedacht für etwa eine
                  halbe Seite (so wie diese Box hier).
            \item \textbf{dsaBoxPortrait}: Hochformat, gedacht für etwa eine
                  Viertelseite (so wie die Box auf der vorherigen Seite).
            \item \textbf{dsaBoxLandscapeSmall}: Querformat, gedacht für etwa
                  eine Achtelseite.
         \end{itemize}
         
         Jede der Umgebungen nimmt als optionalen Parameter ihre Länge.
         Standardmäßig wird \verb|\linewidth| als Länge genommen. Die Höhe
         passt sich dem Inhalt an, das Hintergrundbild wird entsprechend
         verzerrt. Folgende vier Längenvariablen können geändert werden, um
         den Abstand des Inhalts zum Rand anzupassen:
         
         \begin{itemize} \itemsep0em
            \item \verb|\dsaBoxLeftPadding| (Standard: 12pt)
            \item \verb|\dsaBoxRightPadding| (Standard: 14pt)
            \item \verb|\dsaBoxTopPadding| (Standard: 7pt)
            \item \verb|\dsaBoxBottomPadding| (Standard: 14pt)
         \end{itemize}

      \columnbreak
      
         Standardmäßig werden die Boxen nicht in eine
         \texttt{float}-Umgebung gepackt. Das erscheint mir nicht als sinnvoll,
         denn die Boxen sollten schon genau dort erscheinen, wo der Autor sie
         haben möchte. Bildverweise oder ähnliches sind hier wohl eher fehl
         am Platz.
         
         Der Inhalt in den Boxen befindet sich automatisch in einer Minipage,
         das heißt, es können alle üblichen Formatierungstechniken verwendet
         werden. In dieser Box etwa wird der Text mittels einer
         \verb|multicols|-Umgebung zweispaltig gelayoutet.
         
         Beispiel:
         
         \begin{lstlisting}
\begin[0.5\linewidth]{dsaBoxPortrait}
   \subsection*{Informationskasten}
   
   Hier steht etwas interessantes drin.
\end{dsaBoxPortrait}\end{lstlisting}
         
      \end{multicols}
      \vspace{-1pt}
      
   \end{dsaBoxLandscape}
   
   \begin{dsaCharacterSheet}
      \section*{Charakterbögen}
      
      \begin{multicols}{2}
         \begin{dsaSheetBox}[8.5cm]
            \normalfont
            Die Umgebung \textbf{dsaCharacterSheet} startet eine neue Seite,
            auf der ein Charakterbogen, Handout oder ähnliches erstellt werden
            kann. Diese Seite wird mit dem Hintergrundbild \texttt{wallpaper.png}
            im \texttt{fanpaket}-Ordner hinterlegt, sofern dieses Bild vorhanden
            ist (wie auf der ersten Seite beschrieben, ist es nicht Bestandteil
            des Fanpakets). Ansonsten werden die Ränder mit Symbolen verwendet.
            
            Die Seite hat weniger Rand als normale Seiten, um möglichst viel
            Platz verwenden zu können. Mit der Umgebung \textbf{dsaSheetBox}
            können schwarz umrandete Boxen wie diese hier erstellt werden.
            Wird das Hintergrundbild verwendet, wird die Box mit
            halbtransparentem weiß gefüllt. \textbf{dsaSheetBox} nimmt als
            optionales Argument die Breite der Box, deren Standardwert
            \texttt{\textbackslash linewidth} ist.
            
            Auf der gesamten Seite wird standardmäßig die Schriftart für
            Überschriften verwendet, mit \texttt{\textbackslash normalfont} kann zur
            Schriftart für Fließtext gewechselt werden.
         \end{dsaSheetBox}
         
         \begin{dsaSheetBox}[8.5cm]
            \subsection*{Tabellen}
            
            \normalfont
            Charakterbögen beinhalten viele Tabellen. Hier sollte die
            \textbf{tabu}-Umgebung (statt \textbf{tabular}) verwendet werden.
            Hierfür wird folgendes Kommando zur Verfügung gestellt:
            
            \texttt{\textbackslash dsaRow\{<rowheight>\}\{<font>\}\{<content>\}}
            
            Fügt eine Zeile inklusive abschließendem Zeilenumbruch in eine Tabelle ein.
            Sowohl die Höhe der Zeile wie auch die benutzte Schriftart können angepasst
            werden. \texttt{<rowheight>} nimmt übliche Textgrößenparameter wie
            \texttt{\textbackslash scriptsize}. Mit \texttt{<font>} kann die
            Schriftart, -größe usw. für die gesamte Zeile festgelegt werden.
            
            Ein weiteres Kommando, das genutzt werden kann, ist:
            
            \texttt{\textbackslash dsaTextInput[<fontsize>]\{<name>\}\{<width>\}}
            
            Es fügt ein editierbares Textfeld ein. \texttt{<fontsize>} bestimmt die 
            Schriftgröße und steht standardmäßig auf 12 (ohne Einheit angeben!).
            \texttt{<name>} ist der Name des Textfelds (wird nirgends angezeigt,
            es muss nur darauf geachtet werden, dass kein Name doppelt benutzt
            wird). \texttt{<width>} setzt die Breite der Textbox.
         \end{dsaSheetBox}
      
         \columnbreak
         
         \begin{dsaSheetBox}[8.5cm]
            \subsection*{Beispiel: Zwei Boxen nebeneinander}
            \lstinputlisting{dokumentation-snippets/characterSheetBoxes.tex}
         \end{dsaSheetBox}
         
         \begin{dsaSheetBox}[8.5cm]
            \subsection*{Beispiel: Tabelle (Code)}
            \lstinputlisting{dokumentation-snippets/characterSheetTabu.tex}
         \end{dsaSheetBox}
         
         
         \begin{dsaSheetBox}[8.5cm]
            \subsection*{Beispiel: Tabelle (Output)}
            
            \renewcommand{\arraystretch}{1.4}
            \begin{tabu}{p{3.5cm}|p{1.1cm}|p{1.1cm}}
               \dsaRow{\scriptsize}{\scriptsize\normalfont\bfseries\centering}{& Start & Aktuell} \Xhline{2\arrayrulewidth}
               Eigenschaft 1 & \hspace{1pt} \dsaTextInput{E1orig}{0.75cm} & \cellcolor{white} \dsaTextInput{E1cur}{0.75cm} \\ \hline
               Eigenschaft 2 & \hspace{1pt} \dsaTextInput{E2orig}{0.75cm} & \cellcolor{white} \dsaTextInput{E2cur}{0.75cm} \\
            \end{tabu}
         \end{dsaSheetBox}
      \end{multicols}
      
   \end{dsaCharacterSheet}
\end{document}